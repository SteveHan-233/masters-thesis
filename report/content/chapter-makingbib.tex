\chapter{Making the Bibliography with BiB\TeX{}}\label{c:bib}
\index{Making the Bibliography with BiBTeX%
@\emph{Making the Bibliography with BiB\TeX{}}}%

BiB\TeX{} 
\index{BiBTeX@BiB\TeX{}}%
allows one to generate automatically the bibliography 
from a database of bibliographic 
items. You need to do the following:

\begin{enumerate}
\item Create the bibliographic database, 
\index{bibliographic database}%
which is a file whose name ends in \texttt{.bib}. 
\index{.bib@\texttt{.bib}}%
Let us call it \texttt{diss.bib}. Entries in this file are like this:
\begin{verbatim}
@BOOK{knuth:tb,
  author = "Donald K. Knuth",
  title = "The \TeXbook",
  publisher = "Addison-Wesley",
  year = "1984",
}
@TECHREPORT{poorten:sp,
  author = "Alf~J.~van der Poorten",
  title = "Some problems of recurrent interest",
  institution = "School of Mathematics and Physics,
                 Macquarie University",
  address = "North Ryde, Australia 2113",
  number = "81-0037",
  month = "August",
  year = "1981",
}
@ARTICLE{erdos:oap,
 author = "Paul Erd{\"o}s and Paul Turan",
 title = "On a problem in the theory of uniform 
          distribution, {I}", 
 journal = "Indag. Math.",
 volume = "10",
 year = "1948",
 pages = "370--378",
}
\end{verbatim}

\item Include a \cn{bibliographystyle} 
\index{commands!bibliographystyle@\cn{bibliographystyle}}%
command in your \LaTeX{} file, say 

\cn{bibliographystyle\{plain\}} 
and a \cn{bibliography} 
\index{commands!bibliography@\cn{bibliography}}%
command to load the bibliography, 
in this case \cn{bibliography\{diss\}}, at the point of your 
document where the bibliography should be inserted. 

The document at this point will look like this:
\begin{verbatim}
\bibliographystyle{plain}
\bibliography{diss}
\end{verbatim}

\item Run \LaTeX{} on your main file, say \texttt{foo.tex}: 
\texttt{latex foo}. This generates an auxiliary file 
\texttt{foo.aux} with a list of \cn{cite} 
\index{commands!cite@\cn{cite}}
references.

\item Run BiB\TeX{} on your file: \texttt{bibtex foo}. 
BiB\TeX{} reads the auxiliary file, looks up the 
bibliographic database (\texttt{diss.bib}), 
and writes a \texttt{.bbl} 
\index{.bbl@\texttt{.bbl}}%
file with the bibliographic information formated according to
the bibliographic style file (\texttt{.bst}, 
\index{.bst@\texttt{.bst}}%
say \texttt{plain.bst}) 
\index{plain.bst@\texttt{plain.bst}}%
specified.  Messages about resources used and error messages
are written to a \texttt{.blg} 
\index{.blg@\texttt{.blg}}%
file (in the case of this template, disstemplate.blg).

\item Run \LaTeX{} again: \texttt{latex foo}, which now 
reads the \texttt{.bbl} 
\index{.bbl@\texttt{.bbl}}%
reference file.

\item Run \LaTeX{} for a third time: \texttt{latex foo}, 
resolving all references.

\end{enumerate}

This includes all bibliographic items that have been cited 
in the document with a \cn{cite} 
\index{commands!cite@\cn{cite}}%
command. In order to include non cited items in the bibliography,
use the command \cn{nocite}. For example, \cn{nocite\{knuth:tb\}}
anywhere in the document (after \cn{begin\{document\}}) includes 
in the bibliography the item with label \texttt{knuth:tb}. 
In order to include \emph{all} items of the bibliographic 
database, use the command \cn{nocite\{*\}}.
\index{commands!nocite@\cn{nocite}}%
